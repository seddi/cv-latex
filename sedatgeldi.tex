\documentclass[10pt,a4paper]{article}
\usepackage[textwidth=18cm,textheight=25cm]{geometry}
\usepackage[pdftex]{graphicx}
\usepackage{multirow}
% cv-custom.tex
%
% Custom macro commands and packages
% for formatting a Curriculum Vitae.
%
% Author: Matthew Earnshaw <matt@earnshaw.org.uk>
% Inspired by http://www.cv-templates.info/2009/03/professional-cv-latex/

\pagestyle{empty}

% Packages
\usepackage[usenames,dvipsnames]{color} % For custom colours
\usepackage{titlesec} % For custom section headings
\usepackage{mdwlist} % For compact lists
\usepackage[pdftex]{hyperref}
\usepackage{marvosym} % For icons

% Hyperlink colour and style
\definecolor{linkcolour}{rgb}{0,0.2,0.6}
\hypersetup{colorlinks,breaklinks, urlcolor=linkcolour, linkcolor=linkcolour}

% Custom colour
\definecolor{lgray}{gray}{0.4}

% Custom list bullet
\renewcommand{\labelitemi}{$\succ$}

% Header commands
\newcommand{\name}[1]{\LARGE\textbf{#1}}
\newcommand{\address}[1]{\color{lgray}{#1}}
\newcommand{\tel}[1]{\Large\Telefon~\small{#1}}
\newcommand{\email}[1]{\Large\Letter~\href{mailto:#1}{\small{#1}}}
\newcommand{\web}[2]{\Large\Mundus~\href{#1}{\small{#2}}}
	
% Section headings
\titleformat{\section}{\large\scshape\raggedright}{}{0em}{}[\titlerule]
\titlespacing{\section}{0pt}{0.6cm}{5pt}
% Note: Create an environment for sections ?

% Full width tables
\newenvironment{ftabular}[1]
{\begin{tabular*}{0.95\textwidth}{@{\extracolsep{\fill}}#1}}
{\end{tabular*}}

\usepackage[utf8]{inputenc}

\begin{document}
\footnotesize
\fontfamily{ptm}

\section{K{\footnotesize İ}ş{\footnotesize İ}sel B{\footnotesize İ}lg{\footnotesize İ}ler}

\begin{tabular}{l l l}
\vspace{0.5 mm}\\
\multirow{6}{*}{\includegraphics[height=30mm, width=32mm]{sedat.jpg}}
& \name{Sedat Geldi} & Tarih: \small{08/07/2012}\\
\vspace{0.5 mm}\\
& \textbf{Adres :} \address{Aydınpınar köyü cuma mah. Düzce/Merkez} & \tel{0544 7411174}\\
& \textbf{Doğum Tarihi :} \address{01.11.1987} & \email{sedat.geldi@bil.omu.edu.tr}\\
& \textbf{Medeni Hal :} \address{Bekar} & \web{http://seddi.me}{seddi.me}\\
& \textbf{Askerlik Durumu  :} \address{Tecilli 16.06.2014} & \\
& \textbf{Engellilik Durumu :} \address{Hayır} & \\
\end{tabular}

\section{\sc E{\footnotesizeĞ\footnotesize İT\footnotesize İM} B{\footnotesize İLG\footnotesize İLER\footnotesize İ}}
\hspace*{1.6in}\begin{tabular}{lr}
\vspace{0.5 mm}\\
\textbf{Eğitim Seviyesi :} & Üniversite \\
\vspace{0.5 mm}\\
\textbf{Üniversite :} & Samsun 19 Mayıs Üniversitesi \\
\textbf{Fakülte/Enstitü :} & Mühendislik Fakültesi \\
\textbf{Bölüm :} & Bilgisayar Mühendisliği \\
\textbf{Öğrenim Tipi / Dili :} & Türkçe / Örgün Öğretim\\
\textbf{Not Sist. / Mez. Derecesi :} & 74.60 \\
\textbf{Başlama / Mez. Tarihi :} & Eylül - 2008 / Haziran - 2012\\
\vspace{0.5 mm}\\
\textbf{Lise Türü / Bölüm :} & Yabancı Dil Ağırlıklı Lise / Fen\\
\textbf{Öğrenim Tipi / Dili :} & Örgün Öğretim / Türkçe-{\footnotesize İ}ngilizce\\
\textbf{Lise Adı :} & Düzce Lisesi\\
\textbf{Başlama / Mez. Tarihi :} & Eylül -2002 / Haziran-2006\\
\vspace{0.5 mm}\\
\end{tabular}

\underline{\textbf{Yabancı Diller}}
\vspace{0.5 mm}\\
\begin{itemize}
	\item{\textbf{Dil :} ingilizce}
	\begin{itemize}
		\item{\textbf{Genel :} orta}
		\item{\textbf{Okuma :} iyi}
		\item{\textbf{Yazma :} orta}
		\item{\textbf{Konuşma :} başlangıç}
	\end{itemize}		
\end{itemize}

\section{İş Deney{\footnotesize İ}m{\footnotesize İ}}
\begin{ftabular}{r|p{14cm}}
\textsc{Ek{\footnotesize İ}m–2009 Aralık-2010} & \textbf{Uzaktan Eğitim Merkezi - Samsun} \\
\vspace{0.5 mm}\\
 & \textbf{Sektör – Bölüm Pozisyon :} Eğitim - Eğitim - Bilgisayar Mühendisi\\
 & \textbf{İşin Tanımı :} Ebelik lisans tamamlama programında ebelere internet üzerinden ders verilmekte ,
Burada video işleme ve destek bölümünde yer aldım\\

\multicolumn{2}{c}{ } \\ % Spacer 

\textsc{Agustos-2011 Eylül-2011} & \textbf{Tüb{\footnotesize İ}tak UEKAE B{\footnotesize İ}lgem Pardus (Stajer)} \\
\vspace{0.5 mm}\\
 & \textbf{Sektör – Bölüm Pozisyon :} Bilgisayar / BT / Internet - Bilgi Teknolojileri - Bilgisayar Mühendisi\\
 & \textbf{İşin Tanımı :} Bu çalışmada perl dili ile OBS( Open Build Service) sisteminin entegrasyonu yapılmaya çalışılmıştır\\

\multicolumn{2}{c}{ } \\ % Spacer 

\textsc{Temmuz-2010 Agustos-2010} & \textbf{Yeni Hayat Bilişim (Stajer)} \\
\vspace{0.5 mm}\\
 & \textbf{Sektör – Bölüm Pozisyon :} Eğitim - Eğitim - Bilgisayar Mühendisi\\
 & \textbf{İşin Tanımı :} Python modüllerinin araştırılması ve kullanılması çalışmaları yapıldı\\

\end{ftabular}

\section{\sc B{\footnotesize İ}lg{\footnotesize İ}sayar B{\footnotesize İ}lg{\footnotesize İ}s{\footnotesize İ}}

{\bf Programlama dilleri (çok iyi)}\\
\hspace*{0.3in}\begin{tabular}{lrrrr}
\vspace{0.5 mm}\\
	$\bullet$ C &$\bullet$ Python &$\bullet$ Matlab & &\\
\end{tabular}
\vspace{0.5 mm}\\

{\bf Programlama dilleri (iyi)}\\
\hspace*{0.3in}\begin{tabular}{lrrrr}
\vspace{0.5 mm}\\
  $\bullet$ C$ \# $ &$\bullet$ Java & & &\\
\end{tabular}
\vspace{0.5 mm}\\

{\bf Programlama dilleri (orta)}\\
\hspace*{0.3in}\begin{tabular}{lrrrr}
\vspace{0.5 mm}\\
  $\bullet$ Pic Basic & $\bullet$ Bash & & &\\
\end{tabular}
\vspace{0.5 mm}\\

{\bf Programlama dilleri (Başlangıç)}\\
\hspace*{0.3in}\begin{tabular}{lrrrr}
\vspace{0.5 mm}\\
  $\bullet$ Perl & $\bullet$Ruby & $\bullet$Rails & &\\
\end{tabular}
\vspace{0.5 mm}\\

{\bf İşletim sistemleri}\\
\hspace*{0.3in}\begin{tabular}{lrrrr}
\vspace{0.5 mm}\\
  $\bullet$ GNU/Linux ( iyi ) &$\bullet$ Windows\textregistered & & &\\
\vspace{0.5 mm}\\
\end{tabular}

\section{Çalışmalar}
\hspace*{0.1in}\footnotesize{``Akıllı kart sistemleri üzerinde yapmış olduğum kampüs kart masaüstü uygulaması  \web{https://github.com/seddi/Kampuskart}{Kampüskart}``}\\	
\hspace*{0.3in}\footnotesize{``Tüm Çalışmalarımı github adresimde bulabilirsiniz. \web{https://github.com/seddi}{seddi}``}\\


\section{K{\footnotesize URS }S{\footnotesize EM{\footnotesize İ}NERLER} {\footnotesize ve} E{\footnotesize TK{\footnotesize İ}NL{\footnotesize İ}KLER}}
\begin{ftabular}{r|p{14cm}}
\textsc{8-10 mayıs 2012} & \textbf{Yeteneğe Destek Yaratıcı Ekonomiye Destek} \\
\vspace{0.5 mm}\\
 & \textbf{Kurum Adı :}  TTNET\\

\multicolumn{2}{c}{ } \\

\textsc{25-26-27 şubat 2011} & \textbf{Bilmök 7} \\
\vspace{0.5 mm}\\
 & \textbf{Kurum Adı :}  Yeditepe Üniversitesi\\
 
\multicolumn{2}{c}{ } \\

\textsc{16-31 Temmuz 2011} & \textbf{Linux Sistem Yönetimi Yaz Kampı} \\
\vspace{0.5 mm}\\
 & \textbf{Kurum Adı :}  LKD ( Linux Kullanıcıları Derneği)\\
 & \textbf{İçerik :}  \\
 & \hspace*{0.3in} $\bullet$ Komut Satırı (Kabuk) ve Temel Komutlar\\
 & \hspace*{0.3in} $\bullet$ GNU/Linux İşletim Sisteminin Yapısı\\
 & \hspace*{0.3in} $\bullet$ Açılış Sistemi ve Kullanıcı Yönetimi\\
 & \hspace*{0.3in} $\bullet$ Paket Yönetim Sistemi\\
 & \hspace*{0.3in} $\bullet$ Zamanlanmış Görevler\\
 & \hspace*{0.3in} $\bullet$ Sistem Kayıtları\\
 & \hspace*{0.3in} $\bullet$ Temel TCP/IP Bilgisi ve Ağ Yönetimi\\
 & \hspace*{0.3in} $\bullet$ Güvenli Uzaktan Erişim\\
 & \hspace*{0.3in} $\bullet$ Yedekleme\\
 & \hspace*{0.3in} $\bullet$ DNS ve Web sunucusuna Giriş\\
 & \hspace*{0.3in} $\bullet$ Veritabanına Giriş ve MySQL Veritabanı Sunucusu\\
 & \hspace*{0.3in} $\bullet$ WordPress ve Mediawiki ile Web Sitesi Oluşturma\\


 
\multicolumn{2}{c}{ } \\

\textsc{2-3 nisan 2010} & \textbf{Özgür Yazılım Şenliği} \\
\vspace{0.5 mm}\\
 & \textbf{Kurum Adı :}  Bilgi Üniversitesi\\

\multicolumn{2}{c}{ } \\

\textsc{5-6 mayıs 2012} & \textbf{Robot Yarışması} \\
\vspace{0.5 mm}\\
 & \textbf{Kurum Adı :}  Uludağ Üniversitesi\\


\end{ftabular}
\section{E{\footnotesize K} B{\footnotesize {\footnotesize İ}LG{\footnotesize İ}LER}}

Robot kulubü yönetim üyesi (Omü Robotek), Aktif olarak pic programlama görevinde rol aldım
\end{document}
